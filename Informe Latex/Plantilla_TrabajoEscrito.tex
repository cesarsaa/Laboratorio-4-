%%%%%%%%%%%%%%%%%%%%%%%%%%%%%%%%%%%%%%%%%%%%%%%%%%%%%%%%%%%%%%%%%%%%%%%%%%%%%%%%%%%%%%%%%%%%%%%
%Plantilla: para la realizaci�n de informes.
%Curso:     Simulaci�n estad�stica.
%Profesor:  Johann A. Ospina.
%%%%%%%%%%%%%%%%%%%%%%%%%%%%%%%%%%%%%%%%%%%%%%%%%%%%%%%%%%%%%%%%%%%%%%%%%%%%%%%%%%%%%%%%%%%%%%%


%Establece el tipo de documento (art�culo), tama�o de letra (10pt) a una columna.
\documentclass[letterpaper,12pt,onecolumn,titlepage]{article} 
 
 
% Cargar paquetes
\usepackage{verbatim}
\usepackage{mathrsfs}
\usepackage{amsmath}
\usepackage{amssymb}
\usepackage{subfigure}
\usepackage{ucs}
\usepackage[latin1]{inputenc}
\usepackage[spanish]{babel}
\usepackage{fontenc}
\usepackage{graphicx}
\usepackage{anysize}
\usepackage{fancyhdr}
\usepackage[comma,authoryear]{natbib}
\usepackage{url} %paquete para definir url
\usepackage{hyperref}  %hipervinculos

%Estilo de la p�gina
\pagestyle{fancy}

%Establecer el margen
\marginsize{2cm}{2cm}{1cm}{1cm}
\setlength{\headheight}{13.1pt}


% Portada
\title{
    \textbf{Laboratorio N.4}\
    ~\\{Introduccion a Los Metodos Estadisticos}   
    ~\\{Prueba de Hipotesis y Regresion}}
\author{
    {Diana Carolina Arias Sinisterra Cod. 1528008}
 ~\\{Kevin Steven Garcia Chica Cod. 1533173}
 ~\\{Cesar Andres Saavedra Vanegas Cod. 1628466}}

\date{
     \textbf{Universidad Del Valle}\   
    ~\\{Facultad De Ingenieria}
    ~\\{Estadistica}
    ~\\{Diciembre}
    ~\\{2017}}
 
 
 
\decimalpoint %Poner punto decimal
 
\begin{document}
 
% Se aplica el formato a las p�ginas. Se despliegan: portada e �ndices de materias, figuras y tablas
\renewcommand{\listtablename}{}
\renewcommand{\tablename}{Tabla}
\maketitle
\setcounter{page}{2}
\tableofcontents{}
%\thispagestyle{empty}
%\newpage
\listoffigures{}
\listoftables{}

\thispagestyle{empty}

\newpage
\fancyhead{}
\fancyfoot{}
 
% Encabezado y pie de pagina
\lhead{Introduccion a los Metodos Estadisticos}
\lfoot{Universidad Del Valle}
\rfoot{\thepage}

% Estilo de la bibliograf�a
\bibliographystyle{apalike}
 
% Desarrollo de los contenidos del documento
\section{Prueba De Hipotesis}
\subsection{Situaci\'{o}n 1}

\pagebreak \subsection{Situaci\'{o}n 2}
\subsection{Punto A.}
\subsection{Punto B.} 

\pagebreak\subsection{Situaci\'{o}n 3}
\subsection{Punto A.}
\subsection{Punto B.} 

\pagebreak\subsection{Situaci\'{o}n 4}

\pagebreak\subsection{Situaci\'{o}n 5}
\subsection{Punto A.}
\subsection{Punto B.} 
\subsection{Punto C.}

\pagebreak\subsection{Situaci\'{o}n 6}
\subsection{Punto C.}
\subsection{Punto D.} 

\pagebreak\subsection{Situaci\'{o}n 7}
\subsection{Punto A.}
\subsection{Punto B.}

\pagebreak\section{Regresion}
\subsection{Situaci\'{o}n 1}
\subsection{Punto A.}
\subsection{Punto B.}
\subsection{Punto C.}
\subsection{Punto D.}
\subsection{Punto E.}
\subsection{Punto F.}
\subsection{Punto G.}

\pagebreak\subsection{Situaci\'{o}n 2}
\subsection{Punto A.}
\subsection{Punto B.}
\subsection{Punto C.}

\bibliography{Bibliografia}
\end{document}








