%%%%%%%%%%%%%%%%%%%%%%%%%%%%%%%%%%%%%%%%%%%%%%%%%%%%%%%%%%%%%%%%%%%%%%%%%%%%%%%%%%%%%%%%%%%%%%%
%Plantilla: para la realizaci�n de informes.
%Curso:     Simulaci�n estad�stica.
%Profesor:  Johann A. Ospina.
%%%%%%%%%%%%%%%%%%%%%%%%%%%%%%%%%%%%%%%%%%%%%%%%%%%%%%%%%%%%%%%%%%%%%%%%%%%%%%%%%%%%%%%%%%%%%%%


%Establece el tipo de documento (art�culo), tama�o de letra (10pt) a una columna.
\documentclass[letterpaper,12pt,onecolumn,titlepage]{article} 
 
 
% Cargar paquetes
\usepackage{verbatim}
\usepackage{mathrsfs}
\usepackage{amsmath}
\usepackage{amssymb}
\usepackage{subfigure}
\usepackage{ucs}
\usepackage[latin1]{inputenc}
\usepackage[spanish]{babel}
\usepackage{fontenc}
\usepackage{graphicx}
\usepackage{anysize}
\usepackage{fancyhdr}
\usepackage[comma,authoryear]{natbib}
\usepackage{url} %paquete para definir url
\usepackage{hyperref}  %hipervinculos

%Estilo de la p�gina
\pagestyle{fancy}

%Establecer el margen
\marginsize{2cm}{2cm}{1cm}{1cm}
\setlength{\headheight}{13.1pt}


% Portada
\title{
    \textbf{Laboratorio N.4}\
    ~\\{Introduccion a Los Metodos Estadisticos}   
    ~\\{Prueba de Hipotesis y Regresion}}
\author{
    {Diana Carolina Arias Sinisterra Cod. 1528008}
 ~\\{Kevin Steven Garcia Chica Cod. 1533173}
 ~\\{Cesar Andres Saavedra Vanegas Cod. 1628466}}

\date{
     \textbf{Universidad Del Valle}\   
    ~\\{Facultad De Ingenieria}
    ~\\{Estadistica}
    ~\\{Diciembre}
    ~\\{2017}}
 
 
 
\decimalpoint %Poner punto decimal
 
\begin{document}
 
% Se aplica el formato a las p�ginas. Se despliegan: portada e �ndices de materias, figuras y tablas
\renewcommand{\listtablename}{}
\renewcommand{\tablename}{Tabla}
\maketitle
\setcounter{page}{2}
\tableofcontents{}
%\thispagestyle{empty}
%\newpage
\listoffigures{}
\listoftables{}

\thispagestyle{empty}

\newpage
\fancyhead{}
\fancyfoot{}
 
% Encabezado y pie de pagina
\lhead{Introduccion a los Metodos Estadisticos}
\lfoot{Universidad Del Valle}
\rfoot{\thepage}

% Estilo de la bibliograf�a
\bibliographystyle{apalike}
 
% Desarrollo de los contenidos del documento
\section{Prueba De Hipotesis}
\subsection{Situaci\'{o}n 1}

\pagebreak \subsection{Situaci\'{o}n 2}
\subsection{Punto A.}
\subsection{Punto B.} 

\pagebreak\subsection{Situaci\'{o}n 3}
\subsection{Punto A.}
\subsection{Punto B.} 

\pagebreak\subsection{Situaci\'{o}n 4}

\pagebreak\subsection{Situaci\'{o}n 5}
\subsection{Punto A.}
\subsection{Punto B.} 
\subsection{Punto C.}

\pagebreak\subsection{Situaci\'{o}n 6}
\subsection{Punto C.}
\subsection{Punto D.} 

\pagebreak\subsection{Situaci\'{o}n 7}
\subsection{Punto A.}
\subsection{Punto B.}

\pagebreak\section{Regresion}
\subsection{Situaci\'{o}n 1}
\subsection{Punto A.}
\begin{itemize}
\item Reduccion porcentual del total de solidos: El total de s\'{o}lidos disueltos (a menudo abreviado como TDS, del ingl\'{e}s: Total Dissolved Solids) es una medida del contenido combinado de todas las sustancias inorg\'{a}nicas y org\'{a}nicas contenidas en un l\'{i}quido en forma molecular, ionizada o en forma de suspensi\'{o}n micro-granular (sol coloide). 
Los TDS (Total dissolved solids) son la suma de los minerales, sales, metales, cati\'{o}nes o aniones disueltos en el agua. Esto incluye cualquier elemento presente en el agua que no sea (H20) mol\'{e}cula de agua pura y s\'{o}lidos en suspensi\'{o}n. (S\'{o}lidos en suspensi\'{o}n son part\'{i}culas \'{o} sustancias que ni se disuelven ni se asientan en el agua, tales como pulpa de madera.)
En general, la concentraci\'{o}n de s\'{o}lidos disueltos totales es la suma de los cati\'{o}nes (carga positiva) y aniones (cargado negativamente) iones en el agua.
\item Reduccion porcentual de demanda bioquimica de oxigeno: La demanda bioqu\'{i}mica de ox\'{i}geno (DBO) es un par\'{a}metro que mide la cantidad de diox\'{i}geno consumido al degradar la materia org\'{a}nica de una muestra l\'{i}quida.
\end{itemize} 
\pagebreak\subsection{Punto B.}
~\\ \begin{figure}[!h]
    \begin{center}
        \includegraphics[width=10cm]{Figuras/punto1b.png}
        \caption{Gr\'{a}fica de dispersi\'{o}n con recta de ajuste entre las variables X y Y}
        \label{fig:Densidad}
    \end{center}
\end{figure}
~\\ En esta imagen podemos ver que hay una correlaci\'{o}n positiva bastante fuerte, por lo que esperamos que el coeficiente de correlaci\'{o}n sea cercano a 1. Tambi\'{e}n podemos ver que las distancias de los datos o los puntos a la recta de ajuste son considerablemente peque\~{n}as, por lo cu\'{a}l creemos que el $R^2$ es tambi\'{e}n cercano a 1, dici\'{e}ndonos esto, que el modelo ajustado representara en un alto porcentaje la variaci\'{o}n total de Y.
~\\ Procedemos a calcular los dos valores mencionados:
\begin{itemize}
\item Coeficiente de correlaci\'{o}n: $\hat{\rho}=r=\frac{\sum\limits_{i=1}^{n}(x_{i}-\bar{x})(y_{i}-\bar{y})}{\sqrt{\sum\limits_{i=1}^{n}(x_{i}-\bar{x})^2}\cdot\sqrt{\sum\limits_{i=1}^{n}(y_{i}-\bar{y})^2}}$
~\\ $\bar{x}=31.095238$ y $\bar{y}=31.9523809$, entonces:
~\\ $$\hat{\rho}=r=\frac{3202.095238}{\sqrt{3543.809524\cdot\sqrt{3086.952381}}}=\frac{3202.095238}{3307.502267}=0.9681$$
\item Coeficiente de determinaci\'{o}n $R^2$: $R^2=r^2=0.9372$
~\\ Tal y como esperabamos, el coeficiente de correlaci\'{o}n lineal, nos arrojo un resultado de 0.9681 que es bastante alto. Este resultado nos dice que a medida que X aumenta, Y tambien aumenta en una proporci\'{o}n aproximada de 0.9681. Con respecto al $R^2$, tambi\'{e}n nos arroj\'{o} un valor que esperabamos (0.9372) que es bastante alto; este valor nos dice que el $93.72\%$ de la variabilidad total de la variable Y, es explicada por la variable X.
Entonces, seg\'{u}n lo anterior, concluimos que estas dos variables tienen una relaci\'{o}n positiva demasiado fuerte, es decir, cuando aumenta X, Y tambien aumentara, y por el $R^2$ concluimos que X es una buena variable explicativa para Y.
\end{itemize} 
\subsection{Punto C.}
~\\ Para evaluar la asociacion lineal entre las variables X y Y, tendremos que plantear las hipotesis sobre $\rho$ de la siguiente manera:

~\\ $H_{0}$: $\rho=0$ (No hay correlacion lineal)
~\\ $H_{1}$: $\rho\neq 0$ (Hay correlacion lineal)

~\\ $T_{\rho}=\frac{r-\rho}{\sqrt{\frac{1-r^2}{n-2}}}$
~\\ Reemplazando los valores, teniendo en cuenta que por el punto anterior $r=0.9681$, tenemos:
~\\ $T_{\rho}=\frac{0.9681}{\sqrt{\frac{1-(0.9681)^2}{21-2}}}=16.84139$

~\\ Tomando $\alpha=0.05$, $T_{(0.975;19)}=2.093$
~\\ Como $T_{\rho}=16.84139>2.093$, rechazamos $H_{0}$ y concluimos que con una confianza del $95\%$, si existe correlacion lineal entre las dos variables.  
\subsection{Punto D.}
~\\ El modelo ajustado es de la forma: $Y_{i}=\beta_{0}+\beta_{1}X_{i}+e_{i}$
~\\ $$\hat{\beta_{1}}=\frac{\sum\limits_{i=1}^{n}x_{i}y_{i}-n\bar{x}\bar{y}}{\sum\limits_{i=1}^{n}{x_{i}}^2 - n\bar{x}^2}=\frac{24067-21(31.095238)(31.9523809)}{23849-21(31.095238)^2}=\frac{3202.095336}{3543.809648}=0.903574$$

~\\ $$\hat{\beta_{0}}=\bar{y}-\hat{\beta_{1}}\bar{x}=31.9523809-(0.903574\cdot 31.095238)=3.855532$$

~\\ El modelo ajustado queda: $Y_{i}=3.855532+0.903574 X_{i}+ e_{i}$

~\\ \textbf{INTERPRETACI\'{O}N:}
~\\ $\hat{\beta_{0}}=3.855532$ : Sin tener en cuenta la variabilidad de la reducci\'{o}n porcentual del total de s\'{o}lidos, se espera una reducci\'{o}n porcentual de demanda bioqu\'{i}mica de oxigeno de 3.855532.

~\\ $\hat{\beta_{1}}=0.903574$ : Por cada unidad adicional de la reducci\'{o}n porcentual del total de s\'{o}lidos, se espera que la reducci\'{o}n porcentual de demanda bioqu\'{i}mica de oxigeno aumente en promedio en 0.903574 unidades.

\pagebreak\subsection{Punto E.}
~\\ Los indicadores utilizados para evaluar la bondad de ajuste de un modelo son el $\rho$ y el $R^2$, como ya los obtuvimos en el punto b, vamos a interpretarlos.

~\\ $\rho=0.9681$: Este valor nos indica que existe una correlaci\'{o}n positiva bastante fuerte (casi perfecta), es decir, cuando x aumenta, y aumenta casi en la misma proporci\'{o}n. Cuando tenemos una correlaci\'{o}n tan alta entre las dos variables, podemos estar seguros de que el modelo ajustado ser\'{a} un buen modelo ($R^2$ tendiendo a 1) para explicar Y en t\'{e}rminos de X.

~\\ $R^2=0.9372$: Este valor nos dice que el $93.72\%$ de la variabilidad total de variable Y(reducci\'{o}n porcentual de demanda bioqu\'{i}mica de oxigeno) es exlicada por la variable X(reducci\'{o}n porcentual del total de s\'{o}lidos). Lo que en el fondo nos dice que el modelo es bastante bueno, ya que las distancias de los datos o los puntos a la recta de regresi\'{o}n ajustada, son muy peque\~{n}as.

\subsection{Punto F.}
~\\ \textbf{Para $\beta_{0}$:} $\langle \beta_{0} \rangle_{(1-\alpha)\%}=\langle \hat{\beta_{0}}\pm t_{(\frac{\alpha}{2},n-2)}\sqrt{V(\hat{\beta_{0}})} \rangle$

~\\ $V(\hat{\beta_{0}})=\frac{\hat{\sigma^2} \sum\limits_{i=1}^{n}{x_{i}}^2}{n S_{xx}}$

~\\ $\hat{\sigma^2}=\frac{1}{n-2}[S_{yy}-\hat{\beta_{1}}S_{xy}]=\frac{1}{21-2}[3086.952381-(0.903574\cdot 3202.095238)]=\frac{1}{19}(193.6223784)=10.1906515$

~\\ Entonces, reemplazando: $V(\hat{\beta_{0}})=\frac{10.196515\cdot 23849}{21\cdot 3543.809524}=3.265746$

~\\ Ahora, con $\alpha=0.05$:
~\\ $$\langle \beta_{0} \rangle_{0.95\%}=\langle 3.855532 \pm 2.093\cdot \sqrt{3.265746} \rangle$$
~\\ $$(0.073193 ; 7.6378708) $$

~\\ \textbf{Para $\beta_{1}$:} $\langle \beta_{1} \rangle_{(1-\alpha)\%}=\langle \hat{\beta_{1}}\pm t_{(\frac{\alpha}{2},n-2)}\sqrt{V(\hat{\beta_{0}})} \rangle$

~\\ $V(\hat{\beta_{1}})=\frac{\hat{\sigma^2}}{S_{xx}}$
~\\ Ya sabemos que $\hat{\sigma^2}=10.1906515$

~\\ Entonces: $V(\hat{\beta_{1}})=\frac{10.1906515}{3543.809524}=0.00287562$

~\\ Ahora, con $\alpha=0.05$:
~\\ $$\langle \beta_{1} \rangle_{0.95\%}=\langle 0.903574 \pm 2.093\cdot \sqrt{0.00287562} \rangle$$
~\\ $$(0.791337 ; 1.0158107) $$
\subsection{Punto G.}
~\\ \textbf{ESTIMACI\'{O}N PUNTUAL:}
~\\ $E[Y|X]=E[\beta_{0}+\beta_{1}x1+e]=E[\beta_{0}]+E[\beta_{1}x_{1}]+E[e]$, sabemos que $E[e]=0$, por los supuestos del modelo lineal.
~\\ Entonces: $\hat{E[Y|X]}=\hat{\beta_{0}}+\hat{\beta_{1}}x_{1}$
~\\ Por consiguiente: $E[Y|X=30]=\hat{\beta_{0}}+\hat{\beta_{1}}\cdot 30=3.855532+0.903574(30)=30.962752$

~\\ El valor esperado de la reducci\'{o}n de la demanda bioqu\'{i}mica de oxigeno cuando se reduce a $30\%$ el porcentaje total de s\'{o}lidos es de $30.962752\%$.


~\\ \textbf{ESTIMACI\'{O}N POR INTERVALOS:}
~\\ $\langle E[Y|X=30] \rangle_{0.95\%}=\langle \hat{E[Y|X=30]} \pm t_{(\frac{\alpha}{2};n-2)} $

\pagebreak\subsection{Situaci\'{o}n 2}
\subsection{Punto A.}
\subsection{Punto B.}
\subsection{Punto C.}

\bibliography{Bibliografia}
\end{document}








